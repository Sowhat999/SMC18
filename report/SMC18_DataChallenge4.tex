\documentclass{article}
\usepackage{hyperref}
\usepackage{listings}
\pagenumbering{gobble}
\usepackage[english]{babel}
\usepackage[pdftex]{color,graphicx}
\usepackage{enumitem}
\setlist[itemize]{leftmargin=*,noitemsep}
\setlist[enumerate]{leftmargin=*,noitemsep}
\usepackage[T1]{fontenc}
\setlength\parindent{0pt}

\title{SMC 2018 Data Challenge \#4}
\author{Ketan Maheshwari (km0@ornl.gov)}

\begin{document}
\maketitle
\section*{Introduction}

This report presents my response to the SMC 2018 Data Challenge 4--Scientific
Publication Mining. I address the challenge by combining the simplicity of the
classic Linux tools with the power of a modern scalable HPC workflow
platform.

The report describes the tools used, their rationale, solutions to the problems
and a glance at the results. Source code and results may be found at
\texttt{github.com/ketancmaheshwari/SMC18}. The source code for the core
solutions are in files \texttt{src/prob*.awk}. The source code for the HPC
parallelization of the core solutions are in files \texttt{src/runprob*.swift}.
All the results are in the \texttt{results/} directory. A summary of the
results and their description and direct links is in table \ref{tbl:results}.

\section*{Tools used}
\subsection*{Software}
Classic Linux tools such as \texttt{awk}, \texttt{sort}, \texttt{grep},
\texttt{tr}, \texttt{sed} and \texttt{bash} are used. Awk is used for the bulk
of processing.  Syntax of awk programs is known to be terse and hard to read by
some accounts. I have taken special care to make the programs as readable as
possible. ANL's Swift (\texttt{swift-lang.org/Swift-T}) is used to run the awk
programs in parallel over the dataset to radically improve performance.  Swift
uses MPI based communication and load-balancing model to parallelize over
shared as well as distributed memory architectures.

\subsection*{Hardware}
A large-memory (24 T) SGI system with 512-core Intel Xeon (2.5GHz) is used for
all the processing. All the IO is memory (\textit{/dev/shm}) bound.

\subsection*{Rationale}
The following list offers the rationale for the choice of tools especially software.
\begin{enumerate}
\item \texttt{Awk} is concise, fast, and expressive -- especially for text processing applications.
\item SQL based tools and databases are hard to port, parallelize, and maintain.
\item Traditional Linux tools are portable and available on almost all the standard Linux distributions.
\item Alternative tools such as modern Python libraries have speed limitations, portability concerns and considerable learning curve. Some of them are still evolving.
\item Swift is used over other tools such as GNU parallel for its portability to both shared and distributed memory clusters.
\end{enumerate}

\section*{Data}
The original data was in two sets (\textit{aminer} and \textit{mag}) of json
files (total 322 files) each file containing a million records. The data also
consisted of a file with a list of records appearing in both sets. An \texttt{awk} script
(\texttt{src/filterdup.awk}) is used to exclude the duplicate records from the
aminer dataset. As a result, there are over 256 million (256,382,605 to be
exact) unique records to process. The total data size is 329G. Several
fields in the data are \texttt{null}.  Those records were avoided where
relevant. Additionally, records related to non-English publications were
avoided where needed.

\subsection*{Auxiliary data}
In addition to the existing data, I use four lists: 1) A list of cities and their
lat-long coordinates (3,517) with population 100,000 or above 2) A list
of countries (190) 3) A list of world universities and research institutes
(8,984) 4) A list of stop-words to avoid in some of the results (161 words).

\section*{Solutions}
\subsection*{Preprocessing}
The \texttt{jq} (stedolan.github.io/jq) tool is used to transform json data to
tabular format (\texttt{src/json2tabular.sh}).  The json files were converted
to tabular files with 19 original columns and one additional column called
\textit{``num\_authors"} showing the number of authors for a given publication
record. The authors field has a semi-colon separator for multiple authors.
Further curation of tabular data is done by removing extraneous space, square
brackets, escape characters and quotes using \texttt{sed}.

\subsection*{Parallelization and Speedup}
Each of the solution is run in parallel over the 322 data files. This has
resulted in radical speedup of solutions. As mentioned in the
description of the solutions in the sections below, none of the solution has
taken more than an hour of runtime--some taking less than a minute.

\subsection*{Problem1}
\textit{Identify the individual or group of individuals who appear to be the expert in a particular field or sub-field.}

This problem is solved in two ways. First part identifies all the entries with
citations higher than 500 for a given search topic
(\texttt{results/meditation\_highly\_cited.txt}). The second part finds the
names of authors whose names are repeating for queried topic with atleast a
certain number of citations in each entry. This will give a reasonable idea of
who are the expert figures in a given research area. One such result in
\texttt{results/cancer\_research\_topauths.txt} shows authors in cancer
research with more than one publication with at least 1,000 citations.  Along
side is a result of a query for all-time list of most cited papers with a
threshold of 20,000 in \texttt{results/top\_papers.txt}. The parallel
implementation of the solution finishes in \textbf{25 seconds}.

\subsection*{Problem2}
\textit{Identify topics that have been researched across all publications.}

This problem is solved by identifying most frequently appearing words in the
collection. Title, abstract and keywords are parsed and top 1,000 frequently
occuring words across the whole collection is found. Several common words (aka
stop-words) are filtered from the results. At over 23 million, the word
``patients'' has most frequent occurrence. The full list of top 1,000 words
may be found in \texttt{/results/top\_1K\_words\_kw\_abs\_title.txt}. The
target collection of publications may be narrowed down to criterions such as
years range.  The parallel implementation of the task finishes in \textbf{9
minutes}.

\subsection*{Problem3}
\textit{visualize the geographic distribution of the topics in the publications.}

This problem is solved by identifying the author's affiliations for the records
that has search topic in them. The affiliation is searched against
three databases--cities, universities and countries to find out the geographic
locations for that research. The results are aggregated to present a list of
centers for which a given keyword appears most frequently. For cities, the
results are plotted on world map. One such example is shown in
\texttt{results/bird\_research\_cities.png} for the topic ``birds''. The \texttt{results/}
directory contains other similar results such as epilepsy, opioid,
meditation research by universities and by countries. The parallel
implementation of the task finishes in \textbf{25 seconds}.

\subsection*{Problem4}
\textit{Identify how topics have shifted over time.}

This problem may be solved in three distinct ways. The first part processes the
database to find out yearwise occurence of two topics. It generates a
list of years and the number of times \textit{both} topics has occured in a
single publication in that year. For example, the result in
\texttt{results/obesity\_sugar.pdf} (plot courtsey S.  Somnath) shows how the
terms ``obesity'' and ``sugar'' have trended together in publications over the
years.

The second part finds the papers that has highest impact in each year and
extracts the keywords in those papers. The impact is computed by the paper that
is cited the most in that year. The result for this task are in
\texttt{results/yearwise\_trending\_keywords.txt} in the form of year,
keywords, citations triplet.

The third part finds the top 10 most frequently occurring terms each year to
find how the topics get in and out of trend over the years. An animation
showing a bubble plot of words changing between the year 1800 and 2017 may be
found at \texttt{results/freqwordsoveryrs.mkv}. A file list of all the words
may be found in the directory \texttt{results/trending\_words\_by\_year}.

Parallelizing the third part was challenging because it involved a
two-level parallel nested foreach loop. The outer loop iterates over the years
and the inner loop iterates over the input files. The parallel implementation
of the task finishes in \textbf{48 minutes}.

\subsection*{Problem5}
\textit{Given a research proposal, determine whether the proposed work has been
accomplished previously.}

This problem has a simple solution: Find the keywords on a new proposal. If
those keywords appear in an existing publication record, it is a suspect. A
broad list of suspects may be found with logical OR between keywords which
could be narrowed down with logical AND. The keywords may be arbitrarily combined in
logical ORs and ANDs. The results file \texttt{/results/suspects.txt} shows
over 1,400 suspects for an AND combination of keywords:  \textit{battery},
\textit{electronics}, \textit{lithium}, and \textit{energy} from English
language papers. The parallel implementation of the task finishes in \textbf{26
seconds}.

\subsection*{Postprocessing}
Some results were postprocessed for sorting and merging purposes using Linux
\texttt{sort, uniq} commands which took negligible time. Some of the results
obtained were postprocessed for visulization using the \texttt{D3} (d3js.org)
graphics framework and \texttt{ffmpeg} libraries.

%\section*{Results}

\section*{Summary}
In this work I show how the classical Linux tools may be leveraged to solve
modern problems. I show that hundreds of millions of records may be processed
in under a minute at scale using Swift. I show the results that offer insight
into the data without employing any formal sophisticated algorithms.

\begin{table}
\begin{center}
% use packages: array
\begin{tabular}{|l|c|c|}
\hline
\textbf{Soln to} & \textbf{Description} & \textbf{Direct link} \\
\hline
p1 & Top authors for cancer research & \href{https://github.com/ketancmaheshwari/SMC18/blob/master/results/cancer_research_topauths.txt}{link} \\
p1 & Highly cited papers for meditation research & \href{https://github.com/ketancmaheshwari/SMC18/blob/master/results/meditation_highly_cited.txt}{link} \\
p1 & Papers with 20K+ citations & \href{https://github.com/ketancmaheshwari/SMC18/blob/master/results/top_papers.txt}{link} \\
p2 & Top 1K researched topics & \href{https://github.com/ketancmaheshwari/SMC18/blob/master/results/top_1K_words_kw_abs_title.txt}{link} \\
p2 & Top 10 researched topics & \href{https://github.com/ketancmaheshwari/SMC18/blob/master/results/top_10_keywords_by_freq.txt}{link} \\
p3 & HIV research by cities & \href{https://github.com/ketancmaheshwari/SMC18/blob/master/results/HIV_AIDS_Research_Cities.txt
}{link} \\
p3 & bird research at universities & \href{https://github.com/ketancmaheshwari/SMC18/blob/master/results/birds_research_at_univs.txt}{link} \\
p3 & bird research cities on map & \href{https://github.com/ketancmaheshwari/SMC18/blob/master/results/bird_research_cities.png}{link} \\
p4 & Obesity-sugar trend & \href{https://github.com/ketancmaheshwari/SMC18/blob/master/results/obesity_sugar.pdf}{link} \\
p4 & animated evolution of top 10 keywords yearwise & \href{https://github.com/ketancmaheshwari/SMC18/blob/master/results/freqwordsoveryears.mkv}{link} \\
p4 & yearwise trending keywords & \href{https://github.com/ketancmaheshwari/SMC18/blob/master/results/yearwise_trending_keywords.txt}{link} \\
p5 & Possible suspects for proposal with given keywords & \href{https://github.com/ketancmaheshwari/SMC18/blob/master/results/suspects.txt}{link} \\
-- & Best paper's citation network & \href{https://github.com/ketancmaheshwari/SMC18/blob/master/results/best_papers.svg}{link} \\
\hline
\end{tabular}
\end{center}
\caption{Summary of results}
\label{tbl:results}
\end{table}

\section*{Acknowledgements.}
I acknowledge Brian Zachary, Suhas Somnath, Drew Schmidt for their helpful
inputs. I acknowledge CADES for the resources for this project.
\end{document}

